\section{Componentes Utilizados}\label{sec:arquitectura}

Como se ha expuesto en la Secci\'on \ref{sec:intro}, el objetivo del
proyecto de la asignatura es realizar un sistema de medici\'on de
temperatura, humedad y luz a trav\'es de dos placas Arduino Uno. Los
datos recogidos por cada una de las placas ser\'an enviados al tercer
componente del proyecto, una placa Discovery STM32F429, la cual se
encargar\'a de procesarlos y mostrar los datos recibidos por su
pantalla.

El sensor de humedad y temperatura utilizado es el modelo DHT22 y
estar\'a conectado al primer Arduino a trav\'es de una placa de
prototipado. Este Arduino se encargar\'a de recibir tanto los datos de
la humedad y la temperatura, procesarlos como se explica en la
Secci\'on~\ref{subsec:dht22} y envi\'arselos a la Discovery utilizando
un protocolo de comunicaci\'on(Secci\'on~\ref{sec:comunicacion}) desarrollado por el grupo.

El sensor fotovoltaico utilizado es el modelo BH1750 y estar\'a
conectado al segundo Arduino a trav\'es de otra placa de
prototipado. Este se encargar\'a de mendir la luz existente en el
ambiente y se la enviar\'a a la Discovery.

Finalmente la placa Discovery recibir\'a los datos enviados por los
Arduinos a trav\'es de dos de sus UARTs y utilizando el protocolo de
comunicaci\'on desarrollado, los procesar\'a y mostrar\'a el resultado
a trav\'es de su pantalla LCD.

As\'i mismo se utilizar\'an un conjunto de leds para implementar un
sencillo c\'odigo visual que depende de ciertos parametros de
estabilidad que se definan y permitan comprobar de forma gr\'afica el
estado del sistema completo. Este estado de estabilidad depender\'a de
la cantidad de luz y la temperatura ya que el grado de luz que incida
sobre el sistema puede causar el sobrecalentamiento del mismo.

Todo el c\'odigo utilizado para el desarrollo del proyecto puede
encontrarse en \url{https://github.com/tutugordillo/SED}.
% Para este proyecto se han seleccionado dos placas Arduinos UNO y una
% placa 
% Discovery STM32F429 así como un sensor fotovoltaico
% BH1750 GY-302, y un sensor de temperatura y humedad DHT22. Las
% conexiones entre los sensores y los Arduinos se han llevado a cabo
% utilizando placas de prototipado.
\subsection{Arduino UNO}\label{subsec:arduino}
Arduino Uno es una placa con un microcontrolador basado en el
Atmega328 de la marca Atmel. Es desarrollado por la empresa de
hardware libre Arduino.

 Posee 14 pines de entrada/salida digitales (6 de los cuales
pueden usarse como salidas PWM) y 6 pines anal\'ogicos. Estos pines
permiten la conexi\'on con cualquier otro dispositivo que sea capaz de
transimitir o recibir se\~nales entre 0 y 5V. Adem\'as algunos de los
mismos poseen funciones especificas. Entre ellos podemos destacar 
los pines digitales 0 y 1 (Rx y Tx) que son utilizados para la transmisi\'on de
se\~nales TTL a trav\'es de su UART. As\'i mismo los pines 10, 11, 12 y
13 pueden set utilizador para llevar a cabo comunicaciones SPI.

Esta placa es de las m\'as b\'asicas e ideal para comenzar a trabajar
en una primera aproximaci\'on con los SEDS. Es muy utilizada para
realizar el prototipado de sistemas m\'as complejos.

\subsection{Discovery STM32F429I}\label{subsec:discovery}

La placa STM32F429I-Discovery se trata, al igual que en el caso
anterior de una placa de prototipado con un microcontrolador
Cortex-M4. Es desarrollada por la empresa STMicroelectronics. 

En este
caso cuenta con mayor n\'umero de pines comparado con el Arduino
Uno. Posee hasta 21 interfaces de comunicaci\'on entre las que podemos
destacar 4 UARTS y 4 USARTS o 3 interfaces $I^2C$ o 6 interfaces SPI
(por las 4 de Arduino UNO). As\'i mismo cuenta
con una pantala LCD-TFT de 2.3 pulgadas y la posibilidad de trabajar
hasta con 17 timers diferentes. Al igual que en el caso anterior
presenta la posibilidad de trabajar con PWM.

\subsection{M\'odulo BH1750}\label{subsec:bh1750}
El componente BH1750 GY-302 es un sensor fotovoltaico que permite
llevar a cabo mediciones digitales de la intensidad de luz
ambiental. Se trata de una versi\'on mejorada de los sensores
anal\'ogicos de luz formados por fotoresistores LDR, peque\~nas resistencias que
var\'ian en funci\'on de la luz que incide sobre las mismas.

El sensor devuelve un valor medido en Lux. Un Lux es la unidad
derivada del Sistema Internacional de Unidades para el nivel de
iluminaci\'on y equivale a un $lumen/m²$. Las mediciones se pueden
llevar a cabo en dos modos: \emph{modo medici\'on una vez} y \emph{modo
  medici\'on continua}. Adem\'as existen distintas resoluciones (1lx,
0.5lx, 4lx) con
las que llevar a cabo las mediciones y de esta manera ajustar la
precisi\'on con la que se lleva a cabo la medici\'on. As\'i el m\'odulo es capaz de detectar un
m\'inimo de 0.11 lx y un m\'aximo de 100000 lx. 

De la combinaci\'on del modo de medici\'on y la resoluci\'on se
obtienen 6 modos de ejecuci\'on:
\begin{itemize}
\item Continuosly H-Resolution Mode
\item Continuosly H-Resolution Mode2
\item Continuosly L-Resolution Mode
\item One Time H-Resolution Mode
\item One Time H-Resolution Mode2
\item One Time L-Resolution Mode
\end{itemize}

Los modos High(H) trabajan con resoluciones de 1 lx y 0.5 lx 

La comunicaci\'on se lleva a cabo a trav\'es de I2C. En funci\'on de
como se encuentre conectado el pin ADDR del m\'odulo se puede trabajar
con dos direcciones:
\begin{itemize}
\item Si se encuentra conectado al pin A3 o a alimentaci\'on se
  trabajar\'a con la direcci\'on 0x5C.
\item Si se deja sin conectar o conectado a GND la comunicaci\'on I2C
  se llevar\'a a cabo mediante la direcci\'on 0x23.
\end{itemize}

\subsection{M\'odulo DTH22}\label{subsec:dht22}

El m\'odulo DTH22 es un sensor de humedad y temperatura
digital. Normalmente el sensor cuenta con 4 pines, uno para la
alimentaci\'on, otro para tierra, uno de datos y un cuarto que no se
utiliza. En este caso se ha utilizado el modelo AOSONG AM2302 con la
ventaja de que \'unicamente posee los tres pines h\'abiles y posee una
resistencia interna permitiendo su funcionamiento sin necesitar
ning\'un otro componente.

Las mediciones se llevan a cabo en conjuntos de 40 bits donde los 16
primeros hacen referencia a la humedad, los 16 siguientes a la
temperatura y los 8 \'ultimos como \emph{checksum}. Todas las mediciones se
realizan de manera digital. La comunicaci\'on comienza con una se\~nal
del microcontrolador al sensor indicando que comienza la
transmisi\'on. Mientras no se reciba esta se\~nal el sensor se encuentra
en estado de \emph{stand-by}. Tras recibir la se\~nal de comienzo el sensor
responder\'a enviando los 40 bits de datos tras los cuales volver\'a
al estado de \emph{stand-by} hasta que reciba una nueva se\~nal de comienzo de
la transmisi\'on. El proceso completo de comunicaci\'on se lleva a
cabo en unos dos segundos. La transmisi\'on de todos los bits por
parte del sensor al microcontrolador comienza con una se\~nal baja (LOW)
de 50 $\mu$s y posteriormente una se\~nal alta (HIGH) en funci\'on de
cuya longitud el bit transmitido ser\'a un 1 o un 0:
\begin{itemize}
\item Si la se\~nal se transmite durante un tiempo inferior a 30 $\mu$s
  el bit transmitido ser\'a un 0.

\item Si la se\~nal HIGH se transmite durante 70 $\mu$s el bit
  transmitido ser\'a un 1.

\end{itemize}

Para conocer el n\'umero de ciclos transmitidos se hace de forma relativa, es decir,
se mide el n\'umero de ciclos propios de LOW y se compara este valor con
el n\'umero de ciclos de HIGH, de esta forma se puede saber si se ha trasmitido
un 1 o un 0. Adem\'as, durante este c\'alculo se desactivan todas las interrupciones
para que no var\'ien el valor final con ciclos que interfieren en la medici\'on.

\subsection{Leds}\label{subsec:leds}

Los leds utilizados son leds usuales los cuales han de ser conectados
junto a resistencias de 220 ohmios para evitar causar da\~no a las placas a las que
se conecta y asegurar su correcto funcionamiento. Se puede utilizar mediante el env\'io de corriente a
trav\'es de los pines de salida digitales de la placa, provocando que
el led se encienda o se apague. As\'i mismo tambi\'en pueden ser
utilizados a trav\'es de los pines que permitan hacer uso de la
modulaci\'on por ancho de pulso (PWM) consiguiendo de esta forma
controlar la intensidad con la que el led emite luz. 

