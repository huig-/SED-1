\section{Arquitectura Utilizada}\label{sec:arquitectura}

Para este proyecto se han seleccionado dos placas Arduinos UNO y una
Discovery ...[Modelo de la placa] así como un sensor fotovoltaico
BH1750 GY-302, y un sensor de temperatura y humedad DHT22.

\subsection{M\'odulo BH1750}
El componente BH1750 GY-302 es un sensor fotovoltaico que permite
llevar a cabo mediciones digitales de la intensidad de luz
ambiental. Se trata de una versi\'on mejorada de los sensores
anal\'ogicos de luz formados por fotoresistores LDR, pequeñas resistencias que
var\'ian en funci\'on de la luz que incide sobre la misma.

El sensor devuelve un valor medido en Lux. Un Lux es la unidad
derivada del Sistema Internacional de Unidades para el nivel de
iluminaci\'on y equivale a un $lumen/m²$. Las mediciones se pueden
llevar a cabo en dos modos: \emph{modo medici\'on una vez} y \emph{modo
  medici\'on continua}. Adem\'as existen distintas resoluciones (1lx,
0.5lx, 4lx) con
las que llevar a cabo las mediciones y de esta manera ajustar la
precisi\'on de estas. As\'i el m\'odulo es capaz de detectar un
m\'inimo de 0.11 lx y un m\'aximo de 100000 lx. 

La comunicaci\'on se lleva a cabo a través de I2C. En funci\'on de
como se encuentre conectado el pin ADDR del m\'odulo se puede trabajar
con dos direcciones:
\begin{itemize}
\item Si se encuentra conectado al pin A3 o a alimentaci\'on se
  trabajar\'a con la direcci\'on 0x5C.
\item Si se deja sin conectar o conectado a GND la comunicaci\'on I2C
  se llevar\'a a cabo mediante la direcci\'on 0x23.
\end{itemize}



Descripción de las placas y los sensores
