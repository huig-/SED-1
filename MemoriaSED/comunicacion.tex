\section{Protocolo de Comunicaci\'on}\label{sec:comunicacion}

El protocolo de comunicaci\'on utilizado tanto por la Discovery como
por los Arduinos para llevar a cabo la transimis\'on de los datos ha
sido desarrollado por los integrantes del grupo.

La comunicaci\'on se lleva a cabo a trav\'es de las UART de cada uno
de los tres dispositivos y las conexiones pueden ser encontradas en la
secci\'on \ref{sec:mod-intercom}.

Para facilitar el proceso de comunicaci\'on se ha definido una
estructura de datos que contiene el tipo de datos que se env\'ia (luz,
temperatatura o humedad) y su valor num\'erico asociado. De esta forma, el primer byte que se env\'ia indica el tipo de dato, y a continuaci\'on viene el valor num\'erico que es un float puesto que es el tipo m\'as peque\~no que contiene cualquiera de los tres tipos de datos. Adem\'as,
tambi\'en han sido implementadas las funciones de env\'io y
recepci\'on de datos a trav\'es de un puerto serie implementado en la
librer\'ia SoftwareSerial.
