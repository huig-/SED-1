% ;; -*- coding: iso-latin-1; TeX-PDF-mode: t; TeX-master: "main" -*-%



\section{Introducci\'on}\label{sec:intro}
\pagenumbering{arabic}
\pagestyle{headings}


Actualmente cada vez es m\'as com\'un escuchar el t\'ermino
\emph{smart} en nuestra vida cotidiana. Nos estamos dirigiendo a un
mundo cada vez m\'as automatizado y en continua interacci\'on con la
tecnolog\'ia. Este t\'ermino se aplica en distintas \'areas como por
ejemplo la
dom\'otica, los sistemas m\'oviles, e incluso el dise�o. El objetivo
asociado al concepto \emph{samrt} es automatizar distintas acciones
que antes eran llevadas a cabo por el ser humano. Entre ellas podemos destacar
las \emph{smart cities} o \emph{ciudades inteligentes} con las que se
pretende automatizar distintos procesos que permitan mejorar la
calidad de vida en las ciudades y avanzar hacia desarrollos m\'as
sostenibles apoyados en la innovaci\'on y la tecnolog\'ia. Este
concepto tiene tambi\'en asociado en muchas ocasiones una
participaci\'on ciudadana activa.

Un ejemplo lo encontramos en la ciudad de Barcelona, considerada
como una de las principales \emph{smart cities} del mundo. En ella
podemos encontrar sensonres que miden el ruido del tr\'afico, analizan
la calidad del aire, controlan el aparcamiento e incluso act\'uan como
mecanismos de iluminaci\'on inteligente. Adem\'as los peatones
tambi\'en son detectados gracias a sensores inteligentes a trav\'es de
los cuales poder observar la actividad y h\'abitos de los mismos e
identificar las zonas de la ciudad m\'as transitadas.

Adem\'as dentro de Barcelona se puede destacar un proyecto encabezado
por el Fab Lab del Instituto de Arquitectura Avanzada de Catalu�a,
mediante el cual se desarrollo un dispositivo capaz de medir los
niveles de contaminaci\'on y de ruido del entorno. Este dispositivo
tendr\'ia que ser portado por cualquier habitante de Barcelona y
permite tambi\'en compartir los datos recogidos en tiempo real.

Por estas razones, y al ser un t\'ermino cada vez m\'as en auge, se ha elegido como proyecto de la asginatura la simulaci\'on
de una peque�a estaci\'on en la que dos sistemas
independientes(Arduinos Uno)
llevar\'an a cabo mediciones de distintos factores \'ambientales a
trav\'es de los correspondientes sensores
(temperatura y luminosidad en nuestro caso) y posteriormente estos
datos ser\'an enviados a un centro(Discovery STM32f249) donde ser\'an
procesados y se podr\'a visualizar informaci\'on de los mismos.

El proyecto elegido es una simplificaci\'on de posibles usos que se
les puede dar a estos sensores en las \emph{smart cities} de hoy en
d\'ia como se ha expuesto anteriormente en el caso de Barcelona. A\'un
as\'i, existen m\'as ciudades que tienen implantados sistemas que
siguen una pol\'itica de funcionamineto similar. Otro ejemplo de ello es
la ciudad de Bilbao en la que existen medidores de $CO_2$ en distintas
zonas de la ciudad. Estos sensores captan los niveles de $CO_2$ en el
ambiente y posteriormente son enviados a un centro de procesamiento
donde se almacenan y se llevan a cabo diversos
calculos. Una vez los datos han sido procesados, estos son utilizados
por el ayuntamiento para restringir el tr\'afico en diversas zonas de
la ciudad o llevar a cabo pol\'iticas medioambientales en las zonas
m\'as afectadas.


