% ;; -*- coding: iso-latin-1; TeX-PDF-mode: t; TeX-master: "main" -*-%



\section{Introducci\'on}\label{sec:intro}
\pagenumbering{arabic}
\pagestyle{headings}


Hoy en d�a, cada vez es m\'as com\'un escuchar el t\'ermino
\emph{smart} en nuestra vida cotidiana. Nos estamos dirigiendo a un
mundo m\'as automatizado y en continua interacci\'on con la
tecnolog\'ia. Este t\'ermino se aplica en distintas \'areas como por
ejemplo la
dom\'otica, los sistemas m\'oviles, e incluso el dise�o. El objetivo
asociado al concepto \emph{smart} es automatizar distintas acciones
que antes eran llevadas a cabo por el ser humano. 

Las \emph{smart homes} son viviendas automatizadas que aportan
servicios de gesti\'on energ\'etica, seguridad, bienestar y
comunicaci\'on. Muchas de estas casas usan sensores de temperatura
para regular el funcionamiento de la calefacci\'on y de la caldera,
mediante telefon\'ia m\'ovil, Wi-Fi o Ethernet. Tambi\'en controlan
los toldos y persianas el\'ectricas, realizando algunas funciones
repetitivas autom\'aticamente o bien para protegerlos del viento. 

Gracias a este tipo de tecnolog\'ias, tambi\'en se produce un ahorro
energ\'etico debido a la racionalizaci\'on de cargas el\'ectricas y
la gesti\'on de tarifas. Adem\'as, se usan sensores de luz y sol para
regular el uso de las luces de la casa.

Otro de los conceptos de moda relacionados con este t\'ermino es el de
las \emph{smart cities} o \emph{ciudades inteligentes} con las que se
pretende automatizar distintos procesos que permitan mejorar la
calidad de vida en las ciudades y avanzar hacia desarrollos m\'as
sostenibles apoyados en la innovaci\'on y la tecnolog\'ia. Este
concepto tiene tambi\'en asociado, en muchas ocasiones, una
participaci\'on ciudadana activa.

Un ejemplo lo encontramos en la ciudad de Barcelona, considerada
como una de las principales \emph{smart cities} del mundo. En ella
podemos encontrar sensores que miden el ruido del tr\'afico, analizan
la calidad del aire, controlan el aparcamiento e incluso act\'uan como
mecanismos de iluminaci\'on inteligente. Adem\'as los peatones
tambi\'en son detectados gracias a sensores inteligentes a trav\'es de
los cuales poder observar la actividad y h\'abitos de los mismos e
identificar las zonas de la ciudad m\'as transitadas.

Dentro de Barcelona, tambi\'en se puede destacar un proyecto encabezado
por el Fab Lab del Instituto de Arquitectura Avanzada de Catalu�a,
que desarrolla un dispositivo capaz de medir los
niveles de contaminaci\'on y de ruido del entorno. Este dispositivo
tendr\'ia que ser portado por cualquier habitante de Barcelona y
permite tambi\'en compartir los datos recogidos en tiempo real.

No obstante, existen otras ciudades que tienen implantados sistemas que
siguen una pol\'itica de funcionamiento similar: Bilbao tiene
instalados medidores de $CO_2$ en distintas
zonas de la ciudad. Estos sensores captan los niveles de $CO_2$ en el
ambiente y posteriormente son enviados a un centro de procesamiento
donde se almacenan y se llevan a cabo diversos
c\'alculos. Una vez los datos han sido procesados, estos son utilizados
por el Ayuntamiento para restringir el tr\'afico en diversas zonas de
la ciudad o llevar a cabo pol\'iticas medioambientales en las zonas
m\'as afectadas.

Debido al auge de las \emph{smart cities}, se ha elegido como proyecto de la asignatura la simulaci\'on
de una peque�a estaci\'on en la que dos sistemas
independientes (Arduinos Uno)
llevar\'an a cabo mediciones de distintos factores ambientales a
trav\'es de diversos sensores y posteriormente estos
datos ser\'an enviados a un centro de procesamiento de datos
(Discovery STM32F249) donde ser\'an analizados y mostrar\'a la
informaci\'on recibida por los Arduinos.

Los sensores seleccionados, por su aplicabilidad tanto en el \'ambito
de las
\emph{smart homes} como en el de las \emph{smart cities} son: un sensor de
temperatura y humedad, con el que se puede obtener f\'acilmente el
\'indice de calor y un sensor de luz.
El proyecto elegido es una simplificaci\'on de posibles usos que se
les puede dar a estos sensores en las \emph{smart cities} de hoy en
d\'ia como se ha expuesto anteriormente en el caso de Barcelona. 
