\section{Implementaci\'on}\label{sec:imp}
En esta secci\'on se presentan los detalles de la implementaci\'on
llevada a cabo as\'i como determinadas decisiones que se han tomado
exponiendo la raz\'on de las mismas.

Para la implementaci\'on del c\'odigo de los dos Arduinos se ha
utilizado el entorno de desarrollo propio del mismo (Arduino IDE)
mientras que para la programaci\'on de la placa STM32F429I-Discovery
se ha utilizado Eclipse como entorno de desarrollo.

El primer arduino se encargar\'a de medir tanto la temperatura como la
humedad. Para la comunicaci\'on se ha utilizado la librer\'ia
SerialSystem. En este caso se implementa en C la funcionalidad
explicada en la Secci\'on~\ref{subsec:dht22}. Se pueden distinguir
tres archivos distintos, uno con extensi\'on .c con su correspondiente .h donde se
encuentra implementada toda la funcionalidad del sensor y un archivo
con extensi\'on .ino en el que se inicializa el sensor y dentro del
m\'etodo loop se llevan a cabo de forma continua las mediciones tanto
de temperatua como de humedad. Tras las mediciones se utiliza la
librer\'ia SerialSoftware para llevar a cabo el env\'io de los datos a
otro dispositivo a trav\'es de su UART.

El segundo arduino se encargar\'a de la medici\'on fotovoltaica. Al
igual que en el caso anterior se pueden distinguir tres ficheros. Dos
con las extensiones .c y .h respectivamente y uno con la extensi\'on
.ino. En este caso el archivo .c solo tiene 4 m\'etodos. Un m\'etodo
begin donde se inicializa el dispositivo, un m\'etodo configure donde
se configura con uno de los modos explicados anteriormente en la
Secci\'on~\ref{subsec:bh1750} y dos m\'etodos read y write que
implementan la comunicaci\'on a trav\'es de la interfaz $I^�C$ con el
sensor. El m\'etodo read implementa la funcionalidad descrita en la
Secci\'on~\ref{subsec:bh1750}. 

Finalmente los detalles de implementaci\'on sobre la placa
STM32F429I-Discovery est\'an relacionados con la comunicaci\'on de la
misma con los dos Arduinos utilizados y la utilizaci\'on de su
pantalla LCD. En cuanto a la comunicaci\'on con los Arduinos se ha
utilizado dos de las UARTS disponibles por la placa. La STM32F429I
dispone de varias UARTS y USARTS. La diferencia entre las mismas es
que las USARTS ampl\'ian la funcionalidad de las UARTS permitiendo la
comunicaci\'on s\'incrona entre la Discovery y el otro dispositivo
involucrado. En nuestro caso ambas nos permiten
llevar a cabo los objetivos marcados. Para implementar la
comunicaci\'on se ha utilizado el protocolo descrito en la
Secci\'on~\ref{sec:comunicacion}. Para el funcionamiento de las dos
UARTS escogidas (una por Arduino) se ha utilizado la librer\'ia de UARTS
de STM. En primer lugar se lleva a cabo la inicializaci\'on de las
mismas en las que se especifican los atributos de la comunicaci\'on
como son el ancho de banda (9600 baudios en nuestro caso), la paridad
o el n\'umero de bits de finalizaci\'on. Posteriormente comienza un
bucle de espera activa para identificar la recepci\'on de los datos
por parte de cada uno de los arduinos, tras lo cual se procesan los
datos recibidos acorde al protocolo descrito y se procesan los mismos.

En lo referente a la LCD se ha utilizado BSP (Board Support Package),
un middleware de STM que nos permite controlar distintos perif\'ericos
de la placa. Lo que hace BSP es mejorar la legibilidad y mejorar la
interacci\'on entre el programador y las librerias que manejan los
drivers de los distintos perif\'ericos. Si inspeccionamos el co\'digo
existente en BSP referente a la pantalla LCD se puede observar
que utiliza directivas existentes en la librer\'ia de
STM para la pantalla LCD e implementa nuevas directivas que facilitan
la programaci\'on de la pantalla LCD permitiendo por ejemplo mostrar
un String llamando directamente a una directiva a la que hay que
especificarle el n\'umero de l\'inea en la que se quiere que se pinte
y el String. Esta directiva internamente lo que hace es enviar a la
LCD caracter por caracter el String especificado utilizando directivas
de la librer\'ia $stm32f4xx\_hal\_ltdc$ calculando de manera
autom\'atica la posici\'on en la que se dibuja cada caracter.

El controlador de la pantalla LCD soporta 2 contenedores como m\'aximo
en los que dibujar. Estos contenedores se encuentran almacenados en la
memoria SDRAM externa de la placa. El contenido se guarda de forma
matricial de tal forma que se trabaja con una matriz 320x239 (o su
equivalente vectorial), la
resoluci\'on de la pantalla LCD . Cada una de las posiciones de esta
matriz representa un bit y lo que hacen las funciones de BSP es
escribir cada pixel del color indicado con la finalidad de acabar
formando los caracteres espec\'ificos.

La implementaci\'on y c\'odigo del proyecto se encuentra disponible en~\url{https://github.com/tutugordillo/SED} dentro del directorio
\emph{Proyecto Final}.  
